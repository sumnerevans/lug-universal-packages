\documentclass{lug}

\usepackage{fontawesome}
\usepackage{etoolbox}
\usepackage{etoolbox}
\usepackage{textcomp}
\usepackage[nodisplayskipstretch]{setspace}
\usepackage{xspace}

\AtBeginEnvironment{minted}{\singlespacing\fontsize{10}{10}\selectfont}

\makeatletter
\patchcmd{\beamer@sectionintoc}{\vskip1.5em}{\vskip0.5em}{}{}
\makeatother

\newcommand{\pmidg}[1]{\parbox{\widthof{#1}}{#1}}
\newcommand{\splitslide}[4]{
    \noindent
    \begin{minipage}{#1 \textwidth - #2 }
        #3
    \end{minipage}%
    \hspace{ \dimexpr #2 * 2 \relax }%
    \begin{minipage}{\textwidth - #1 \textwidth - #2 }
        #4
    \end{minipage}
}
\setbeamerfont{footnote}{size=\tiny}

\title{Universal Packages}
\author{Sumner Evans and Robby Zampino}
\institute{Mines Linux Users Group}

\begin{document}

\section{Introduction}

\begin{frame}{What are packages?}
    \begin{center}
        A \textbf{package} is an archive containing a collection of executable
        files or source code, along with metadata, which represent a computer
        program.
    \end{center}
\end{frame}

\begin{frame}{What is a package format?}
    \begin{center}
        A \textbf{package format} is an organizational structure for delivering
        packages to users.
    \end{center}
\end{frame}

\begin{frame}{Why do we need package formats?}
    \begin{itemize}
        \item They provide a common way to bundle executables, libraries,
            assets, etc.\ for deployment on user machines.
        \item They provide metadata about programs for use in package managers.
        \item It would suck if we had to go find the source code for every
            single program we want to use and compile from
            source.\footnote[frame]{Actually, some package formats do require
            compilation from source (for example some AUR packages) but at least
            it helps automate this process.}
    \end{itemize}
\end{frame}

\section{Appimage}
\section{snapd}
\section{flatpak}
\section{nix}

\section{Love to Hate Them}

\begin{frame}{Proprietary enterprise apps are coming to Linux}
    % TODO
    \begin{itemize}
        \item Canonical snap sprint
    \end{itemize}
    Currently, when enterprises want to target a platform, they see this:
    \begin{enumerate}[leftmargini=*]
        \item[macOS] \texttt{.dmg}
        \item[Windows] \texttt{.exe}
        \item[Android] \texttt{.apk}
        \item[Linux] \texttt{.deb}, \texttt{.rpm}, \texttt{PKGBUILD}, \dots
    \end{enumerate}
\end{frame}

% TODO
\begin{frame}[standout]
    \Huge
    Live Demo
\end{frame}

\begin{frame}[standout]
    \Huge
    Questions?
\end{frame}

\begin{frame}{Resources}
    \begin{itemize}
        \item \url{https://}
    \end{itemize}
\end{frame}

\end{document}
