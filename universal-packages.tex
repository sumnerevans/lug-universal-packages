\documentclass{lug}

\usepackage{fontawesome}
\usepackage{etoolbox}
\usepackage{etoolbox}
\usepackage{textcomp}
\usepackage[nodisplayskipstretch]{setspace}
\usepackage{xspace}

\AtBeginEnvironment{minted}{\singlespacing\fontsize{10}{10}\selectfont}

\makeatletter
\patchcmd{\beamer@sectionintoc}{\vskip1.5em}{\vskip0.5em}{}{}
\makeatother

\newcommand{\pmidg}[1]{\parbox{\widthof{#1}}{#1}}
\newcommand{\splitslide}[4]{
    \noindent
    \begin{minipage}{#1 \textwidth - #2 }
        #3
    \end{minipage}%
    \hspace{ \dimexpr #2 * 2 \relax }%
    \begin{minipage}{\textwidth - #1 \textwidth - #2 }
        #4
    \end{minipage}
}

\title{Filesystems}
\author{Sumner Evans and Sam Sartor}
\institute{Mines Linux Users Group}

\begin{document}

\section{Introduction}

\begin{frame}{What are Filesystems?}
    \begin{itemize}
        \item Filesystems manage the storage and retrieval of files from storage
            media.
        \item Filesystems are an abstraction layer between storage media (SSDs,
            HDDs, disk drives, even tape drives).
        \item Filesystems exist on \textit{partitions}, physically contiguous
            segments of the disk.
    \end{itemize}
\end{frame}

\begin{frame}{Filesystems are Responsible for\ldots}
    \begin{itemize}
        \item \textbf{Space management:} filesystems allocate and manage space
            in discrete chunks. Filesystems must keep track of what data is
            stored at each chunk.
        \item \textbf{Filenames:} identify a storage location in the file
            system. Can be case sensitive (ext4) or case insensitive (HFS,
            NTFS).
        \item \textbf{Directories (folders):} group files into separate
            collections. Modern filesystems allow arbitrary nesting of
            directories.
        \item \textbf{Metadata:} filesystems store book-keeping information
            about their contents (e.g. file sizes, last accessed date, owner and
            permissions, etc.).
        \item \textbf{Access Control:} prevent unauthorized access to files on
            disk.
        \item \textbf{Data Integrity:} filesystems must be resilient to failure,
            some are better at this than others.
    \end{itemize}
\end{frame}

\section{A Few Ancient Filesystems}
\begin{frame}{The First Filesystems}
    \splitslide{0.60}{1em}{
        The filesystem was originally thought of as part of the operating
        system.

        One of the first filesystems that had a name was DECTape. DECTape stored
        184 kilobytes (kilo, not mega) of data per tape on the PDP-8.
    }{}
\end{frame}

\begin{frame}{CP/M $\rightarrow$ FAT}
    Gary Kildall invented CP/M, an OS with a portable filesystem.

    Kildall did not port his system to 16-bits, and Tim Patterson created a
    clone of CP/M for his new operating system: QDOS. Microsoft bought QDOS from
    Patterson, and the filesystem was given the name FAT. This first version of
    FAT was called FAT-12 because it used a 12-bit number to count the clusters.

    $2^{12} = 4096$ and given that each cluster could be up to 8KB, so FAT-12
    volumes had a maximum size of 32MB.

    FAT-12 was superseded by FAT-16 and FAT-32. One limitation was that file
    paths could only be 255 characters long.
\end{frame}

\section{Current Filesystems}

\section{Linux}
\begin{frame}{ext4}
    \texttt{ext4} is the successor to \texttt{ext3} and is the default
    filesystem for many Linux distributions.

    \textbf{Pros}
    \begin{itemize}
        \item It's everywhere. Everyone and their cat use it.
        \item It can support volumes with sizes up to 1 exbibyte (EiB) and files
            with sizes up to 16 tebibytes (TiB).
        \item Backwards compatible with \texttt{ext2} and \texttt{ext3}.
    \end{itemize}

    \textbf{Cons}
    \begin{itemize}
        \item It does not honor the ``secure deletion'' file attribute, which is
            supposed to cause overwriting of files upon deletion
        \item It's boring. It's not new and fancy. It's the tried and true file
            system.
    \end{itemize}
\end{frame}

\section{Windows \& macOS}
\begin{frame}{NTFS}
    New Technology File System (\textbf{NTFS}) is the crap that Windows runs on.

    \textbf{Pros}
    \begin{itemize}
        \item It works on Windows (not sure if this is a pro or not\ldots)
    \end{itemize}

    \textbf{Cons}
    \begin{itemize}
        \item Filenames are not case sensitive and are limited to 255 UTF-8 code
            units
        \item Can resize (post Windows Vista) but cannot move sectors around to
            do so
        \item File paths are limited 32,000 characters
        \item Much less resilient to failure than other filesystems.
    \end{itemize}
\end{frame}

\begin{frame}{HFS and HFS+}
    Apple has made a ton of filesystems with varying degrees of terribleness.
    \begin{itemize}
        \item \textbf{HFS:} Hierarchical File System --- Introduced in 1985 with
            the first Apple computer with a hard drive. Had a limitation of
            65,535 files and every file had to take up at least 1 / 65,535th of
            the disk.

        \item \textbf{HFS+:} Released in 1998 to fix some of the issues with
            HFS. the core of the filesystem uses case-insensitive NFD Unicode
            strings, which led Linus Torvalds to say that ``HFS+ is probably the
            worst file-system ever''.
    \end{itemize}
\end{frame}

\begin{frame}{APFS}
    \textbf{APFS:} Apple Filesystem --- Introduced in June 2016 to replace HFS+
    and is optimized for SSDs. It fixes some of the problems of HFS+. Basically
    it replicates the work of other modern filesystems which are actually
    maintained by large communities.

    Apple forcibly upgraded all computers to APFS in macOS High Sierra. So sorry
    if your data was corrupted. Let me introduce you to an operating system
    where you have a choice to change you filesystem or not: Linux.
\end{frame}

\section{Flashdrives}
\begin{frame}{FAT32}
    \texttt{FAT} is an ancient family of filesystems, going back to floppy disks
    in 1977. \texttt{FAT32} allows for largish files (up to 4GiB) and storage
    devices. It is mainly used on portable storage devices such as flash drives.

    \textbf{Pros}
    \begin{itemize}
        \item Works anywhere
        \item Works on anything
    \end{itemize}

    \textbf{Cons}
    \begin{itemize}
        \item Annoying size limitations
        \item Very inflexible
    \end{itemize}
\end{frame}

\begin{frame}{exFAT}
    \texttt{exFAT} is the latest variation of the \texttt{FAT} family. Although
    it is far from being fully featured, it is a significant improvement on
    \texttt{FAT32}.

    \textbf{Pros}
    \begin{itemize}
        \item Newer
        \item Supports larger files size
        \item Supports larger storage devices
    \end{itemize}

    \textbf{Cons}
    \begin{itemize}
        \item No journaling
        \item Not as ubiquitus
    \end{itemize}
\end{frame}

\section{Alternative Filesystems}
\begin{frame}{Btrfs}
    \textbf{B}-\textbf{tr}ee \textbf{f}ile \textbf{s}ystem (Btrfs) pronounced
    ``Butter FS'' or ``better FS'' or ``b-tree FS'' was developed starting in
    2007 by Oracle.

    \textbf{Pros}
    \begin{itemize}
        \item Copy on Write
        \item Mostly self-healing
        \item Can convert from \texttt{ext*} to Btrfs
    \end{itemize}

    \textbf{Cons}
    \begin{itemize}
        \item It's being deprecated by Oracle. RHEL 7.4 includes it, but they
            are transitioning away. The SUSE project will still use and maintain
            it.
    \end{itemize}
\end{frame}

% \begin{frame}{XFS}

% \end{frame}

\begin{frame}{ZFS}
\texttt{ZFS} is file system designed by Sun Microsystems for long-term,
performant, and reliable data storage.

\textbf{Pros}\begin{itemize}
    \item Corruption detection and self-healing
    \item Integrates well with RAID
    \item Builtin shapshotting and rollback
    \item Highly configurable
\end{itemize}

\textbf{Cons}\begin{itemize}
    \item Not for day-to-day use
    \item Main fork is closed source
\end{itemize}
\end{frame}

\begin{frame}{TFS}
\texttt{TFS} is a work-in-progress filesystem for the Redox operating system.
Intended as a modern alternative to ZFS, the feature list is mouth-watering.

\textbf{Pros}\begin{itemize}
    \item Concurrent \& non-blocking
    \item Lightweight full-disk compression
    \item Zero-overhead revision history
    \item Automatic corruption detection
    \item $O(1)$ recursive directory copies
    \item Designed for solid state drives
    \item Perfectly resilient to sudden power loss
\end{itemize}

\textbf{Cons}\begin{itemize}
    \item Not implemented yet
\end{itemize}
\end{frame}

\section{Network Filesystems}
\begin{frame}{What is a network filesystem?}
\begin{center}
    You can access remote storage devices over the internet using a
    \emph{network filesystem}.
\end{center}
\end{frame}

\begin{frame}{NFS}
    \texttt{NFS} is a common network filesystem protocol for *Nix.

    \textbf{Pros}
    \begin{itemize}
        \item Fast
        \item Mature
        \item Cross platform
    \end{itemize}

    \textbf{Cons}
    \begin{itemize}
        \item Requires setup on both host and client
        \item Server is (relatively) complex to setup
    \end{itemize}
\end{frame}

\begin{frame}{Samba}
\texttt{Samba} includes a network filesystem that interpolates between windows
network drives and *Nix systems. 

\textbf{Pros}
\begin{itemize}
    \item Talk to Windows networks without having to use them
\end{itemize}

\textbf{Cons}
\begin{itemize}
    \item Windows
\end{itemize}
\end{frame}

\section{Virtual Filesystems}
\begin{frame}{What is a virtual filesystem?}
\begin{center}
    A \emph{virtual filesystem} is an abstraction layer that takes in some other
    source of data and represents it as a mountable structure of files and
    directories.

    Colloquially, any filesystem that is associated with a physical disk is
    called a virtual filesystem. These might still be backed by other kind of
    storage, or they might be purely procedural.
\end{center}
\end{frame}

\begin{frame}{tmpfs}
    \texttt{tmpfs} is a filesystem stored in RAM. It appears as a mounted
    filesystem, but all data is stored in volatile memory. By default, the
    \texttt{/tmp} directory is a \texttt{tmpfs}.

    \textbf{Pros}
    \begin{itemize}
        \item Useful for storing temporary files such as downloads and program
            files
        \item Saves unnecessary disk I/O
        \item Safe storage for decrypted data, since files are never written to
            disk
    \end{itemize}

    \textbf{Cons}
    \begin{itemize}
        \item Everything in a \texttt{tmpfs} will be ``deleted'' on reboot.
    \end{itemize}
\end{frame}

\begin{frame}{procfs}
    In *Nix, everything is a file. This includes things such as process
    information. The way to access this data is through the \texttt{proc}
    filesystem which provides a convenient and standardized method for
    dynamically accessing process data held in the kernel.

    In Linux, \texttt{procfs} contains more than just process information
    including memory information, network utilization statistics, etc.
\end{frame}

\begin{frame}{FUSE}
Filesystem in Userspace (FUSE) is an interface for creating filesystems
without writing any kernel-level code, which makes it incredibly useful for
creating virtual filesystems. It is an available in Linux, FreeBSD, OpenBSD,
NetBSD, OpenSolaris, Minix 3, Android, and macOS.

\texttt{libfuse}\ldots \begin{itemize}
    \item is a C library
    \item provides a high-level interface for FUSE
    \item makes creating new filesystems really easy
    \item has bindings for Python, Rust, etc
\end{itemize}
\end{frame}

\begin{frame}{sshfs}
\texttt{sshfs} is a network filesystem implemented through libfuse. You can
use it to mount remote directories via ssh.

\textbf{Pros}\begin{itemize}
    \item Very easy \& quick to setup (one command)
    \item Remote machine only needs ssh installed
\end{itemize}

\textbf{Cons}\begin{itemize}
    \item Intended to be temporary
    \item Generally slower than NFS
\end{itemize}
\end{frame}

\begin{frame}[standout]
    \Huge
    Questions?
\end{frame}

\begin{frame}{References}
    \begin{itemize}
        \item \url{https://en.wikipedia.org/wiki/File_system}
        \item \url{http://www.tldp.org/LDP/sag/html/filesystems.html}
        \item \url{https://arstechnica.com/gadgets/2008/03/past-present-future-file-systems/2/}
    \end{itemize}
\end{frame}

\end{document}
