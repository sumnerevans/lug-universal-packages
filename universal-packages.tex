\documentclass{lug}

\usepackage{fontawesome}
\usepackage{etoolbox}
\usepackage{etoolbox}
\usepackage{textcomp}
\usepackage[nodisplayskipstretch]{setspace}
\usepackage{xspace}
\usepackage{enumitem}

\setitemize{label=\usebeamerfont*{itemize item}%
  \usebeamercolor[fg]{itemize item}
  \usebeamertemplate{itemize item}}

\AtBeginEnvironment{minted}{\singlespacing\fontsize{10}{10}\selectfont}

\makeatletter
\patchcmd{\beamer@sectionintoc}{\vskip1.5em}{\vskip0.5em}{}{}
\makeatother

\newcommand{\pmidg}[1]{\parbox{\widthof{#1}}{#1}}
\newcommand{\splitslide}[4]{%
    \noindent
    \begin{minipage}{#1 \textwidth - #2 }
        #3
    \end{minipage}%
    \hspace{ \dimexpr #2 * 2 \relax }%
    \begin{minipage}{\textwidth - #1 \textwidth - #2 }
        #4
    \end{minipage}
}
\setbeamerfont{footnote}{size=\tiny}

\title{Universal Packages}
\author{Sumner Evans and Robby Zampino}
\institute{Mines Linux Users Group}

\begin{document}

\section{Introduction}

\begin{frame}{What are packages?}
    \begin{center}
        A \textbf{package} is an archive containing a collection of executable
        files or source code, along with metadata, which represent a computer
        program.
    \end{center}
\end{frame}

\begin{frame}{What is a package format?}
    \begin{center}
        A \textbf{package format} is an organizational structure for delivering
        packages to users.
    \end{center}
\end{frame}

\begin{frame}{Why do we need package formats?}
    \begin{itemize}
        \item They provide a common way to bundle executables, libraries,
            assets, etc.\ for deployment on user machines.
        \item They provide metadata about programs for use in package managers.
        \item It would suck if we had to go find the source code for every
            single program we want to use and compile from
            source.\footnote[frame]{Actually, some package formats do require
            compilation from source (for example some AUR packages) but at least
            it helps automate this process.}
    \end{itemize}
\end{frame}

\begin{frame}{A bit of history}
    \begin{enumerate}[leftmargin=2cm]
        \item[1994] \texttt{dpkg} --- the package format behind \texttt{apt} and
            \texttt{apt-get}.  Used by Debian-based systems.
        \item[1997] \texttt{RPM} --- the package format behind \texttt{yum} and
            \texttt{dnf}. Used by RHEL-like systems.
        \item[2002] \texttt{pacman} --- the package manager for Arch Linux. It
            just uses \texttt{tar} files.
        \item[2004] \texttt{klik}/\texttt{PortableLinuxApps}
            (2011)/\texttt{AppImage} (2013) --- a package format built to be
            Linux-distro agnostic.
        \item[2006] \texttt{nix} --- a purely functional package format.
            Primarily used by NixOS.
        \item[June 2016] \texttt{snapd} --- the Canonical-backed universal
            package format is ported to a wide range of Linux distros.
        \item[June 2016] \texttt{Flatpak} --- the Red Hat-backed universal
            package format becomes generally available.
    \end{enumerate}
\end{frame}

\section{Universal Package Formats}
\begin{frame}{Common objectives}
    \begin{itemize}
        \item Linux distro agnosticism
        \item Solve the ``dependency hell''
        \item Create a ``single'' deployment target for all of Linux
    \end{itemize}
\end{frame}

\section{AppImage}
\begin{frame}{Why is AppImage cool?}
    \begin{itemize}
        \item \textbf{AppImage does not require installation.} The AppImage file
            is just its compressed image that is mounted with FUSE when it runs.
        \item \textbf{AppImage does not require root permission.} The
            application is run as the user and the base system is left
            untouched.
        \item \textbf{The AppImage itself is executable.} Just \texttt{chmod +x}
            the \texttt{.AppImage} file and run.
        \item \textbf{Linus says so}
            \begin{quote}
                ``This is just very cool.''\\
                \hspace*{0.5in}\textasciitilde\ Linus Torvalds
            \end{quote}
    \end{itemize}
\end{frame}

\begin{frame}[standout]
    \Huge
    Live Demo: Running an AppImage
\end{frame}

\section{snapd}

\section{flatpak}

\section{nix}


\section{Love to Hate Them}

\begin{frame}{Proprietary enterprise applications are coming to Linux}
    Currently, when enterprises want to make a cross-platform application, they
    see this:
    \begin{enumerate}[leftmargin=2cm]
        \item[macOS] \texttt{.dmg}
        \item[Windows] \texttt{.exe}
        \item[Linux] \texttt{.deb} and \texttt{.rpm} and \texttt{PKGBUILD} and
            \dots then deal with the dependency hell
    \end{enumerate}

    However, when companies like Canonical come in and say ``just target
    snaps'', all of a sudden, it may tip the scale at enterprises for them to
    start targeting Linux. If they create a snap, then they capture all of the
    Linux market, not just the subset that uses a particular format.
\end{frame}

\begin{frame}{Pros and cons}
    \textbf{Pros}
    \begin{itemize}
        \item More application availability.
    \end{itemize}

    \textbf{Cons}
    \begin{itemize}
        \item The applications are going to be crap. Electron, enterprise crap.
    \end{itemize}
\end{frame}

% TODO
\begin{frame}[standout]
    \Huge
    Live Demo
\end{frame}

\begin{frame}[standout]
    \Huge
    Questions?
\end{frame}

\begin{frame}{Resources}
    \begin{itemize}
        \item \url{https://}
    \end{itemize}
\end{frame}

\end{document}
